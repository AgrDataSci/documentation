% Options for packages loaded elsewhere
\PassOptionsToPackage{unicode}{hyperref}
\PassOptionsToPackage{hyphens}{url}
\documentclass[
]{article}
\usepackage{xcolor}
\usepackage[margin=1in]{geometry}
\usepackage{amsmath,amssymb}
\setcounter{secnumdepth}{-\maxdimen} % remove section numbering
\usepackage{iftex}
\ifPDFTeX
  \usepackage[T1]{fontenc}
  \usepackage[utf8]{inputenc}
  \usepackage{textcomp} % provide euro and other symbols
\else % if luatex or xetex
  \usepackage{unicode-math} % this also loads fontspec
  \defaultfontfeatures{Scale=MatchLowercase}
  \defaultfontfeatures[\rmfamily]{Ligatures=TeX,Scale=1}
\fi
\usepackage{lmodern}
\ifPDFTeX\else
  % xetex/luatex font selection
\fi
% Use upquote if available, for straight quotes in verbatim environments
\IfFileExists{upquote.sty}{\usepackage{upquote}}{}
\IfFileExists{microtype.sty}{% use microtype if available
  \usepackage[]{microtype}
  \UseMicrotypeSet[protrusion]{basicmath} % disable protrusion for tt fonts
}{}
\makeatletter
\@ifundefined{KOMAClassName}{% if non-KOMA class
  \IfFileExists{parskip.sty}{%
    \usepackage{parskip}
  }{% else
    \setlength{\parindent}{0pt}
    \setlength{\parskip}{6pt plus 2pt minus 1pt}}
}{% if KOMA class
  \KOMAoptions{parskip=half}}
\makeatother
\usepackage{graphicx}
\makeatletter
\newsavebox\pandoc@box
\newcommand*\pandocbounded[1]{% scales image to fit in text height/width
  \sbox\pandoc@box{#1}%
  \Gscale@div\@tempa{\textheight}{\dimexpr\ht\pandoc@box+\dp\pandoc@box\relax}%
  \Gscale@div\@tempb{\linewidth}{\wd\pandoc@box}%
  \ifdim\@tempb\p@<\@tempa\p@\let\@tempa\@tempb\fi% select the smaller of both
  \ifdim\@tempa\p@<\p@\scalebox{\@tempa}{\usebox\pandoc@box}%
  \else\usebox{\pandoc@box}%
  \fi%
}
% Set default figure placement to htbp
\def\fps@figure{htbp}
\makeatother
\setlength{\emergencystretch}{3em} % prevent overfull lines
\providecommand{\tightlist}{%
  \setlength{\itemsep}{0pt}\setlength{\parskip}{0pt}}
\usepackage{bookmark}
\IfFileExists{xurl.sty}{\usepackage{xurl}}{} % add URL line breaks if available
\urlstyle{same}
\hypersetup{
  hidelinks,
  pdfcreator={LaTeX via pandoc}}

\author{}
\date{\vspace{-2.5em}}

\begin{document}

\section{Tricot Resources}\label{tricot-resources}

\subsection{\texorpdfstring{\hyperref[publications]{Publications}}{Publications}}\label{publications}

\subsection{\texorpdfstring{\hyperref[videos]{Training
Videos}}{Training Videos}}\label{training-videos}

\subsection{\texorpdfstring{\hyperref[costing]{Costing
Templates}}{Costing Templates}}\label{costing-templates}

\subsection{\texorpdfstring{\hyperref[support_material]{Supporting
Material}}{Supporting Material}}\label{supporting-material}

\subsection{Publications}\label{publicatons}

\subsubsection{Guide}\label{guide}

Jacob van Etten, Rhys Manners, Jonathan Steinke, Elsa Matthus, Kauê de
Sousa. 2020. The tricot approach. Guide for large-scale participatory
experiments. Rome (Italy): Alliance of Bioversity International and
CIAT. \url{https://hdl.handle.net/10568/109942}

This is a short, full-colour guide intended for practitioners who are
not yet familiar with tricot. It explains the rationale of tricot and
gives an overview of the experimental cycle.

\subsubsection{Main publications}\label{main-publications}

All publications about the tricot approach are free and open access.

\begin{itemize}
\tightlist
\item
  Jonathan Steinke, Jacob van Etten and Pablo Mejía-Zelan. 2017. The
  accuracy of farmer-generated data in an agricultural citizen science
  methodology. Agronomy for Sustainable Development 37: 32.
  \url{https://doi.org/10.1007/s13593-017-0441-y}
\end{itemize}

The above paper shows that farmers provide accurate data in tricot
trials. Their rankings converge with expert rankings for four traits.
The variation between farmers still allows for accurate overall ranking
of the varieties.

\begin{itemize}
\tightlist
\item
  Eskender Beza, Jonathan Steinke, Jacob van Etten et al.~2017. What are
  the prospects for large-N citizen science in agriculture? Evidence
  from three continents on motivation and mobile telephone use of
  resource-poor farmers participating in ``tricot'' crop research
  trials. PLoS ONE 12(5): e0175700.
  \url{https://doi.org/10.1371/journal.pone.0175700}
\end{itemize}

This paper investigates the motivation of farmers who participate in
tricot trials across three contrasting contexts: Honduras, Ethiopia and
India. Farmers are motivated by a wide range of reasons, including
intrinsic and extrinsic factors. They do not see it as a pastime, but
also do not expect monetary compensation. They expect technical
information and access to seeds as reward of their participation.

\begin{itemize}
\tightlist
\item
  Jacob van Etten, et al.~2019. First experiences with a novel farmer
  citizen science approach: Crowdsourcing participatory variety
  selection through on-farm triadic comparisons of technologies
  (tricot). Experimental Agriculture, 55(S1).
  \url{https://doi.org/10.1017/S0014479716000739}
\end{itemize}

This paper provides an explanation of the tricot approach, how it
compares to previous approaches, and some first applications. Note that
it uses the Bradley-Terry model, which was replaced by the Plackett-Luce
model in later publications.

\begin{itemize}
\tightlist
\item
  Jacob van Etten, Kauê de Sousa, {[}\ldots{]} Jonathan Steinke. 2019.
  Crop variety management for climate adaptation supported by citizen
  science. PNAS 116(10): 4194-4199.
  \url{https://doi.org/10.1073/pnas.1813720116}
\end{itemize}

This paper describes the application of large tricot trials in
Nicaragua, Ethiopia and India. It demonstrates the potential of tricot
to generate insights into variety adaptation, recommend adapted
varieties, and aid smallholder farmers in responding to climate change.
It is the first large-scale application of climate analysis on tricot
data.

\begin{itemize}
\tightlist
\item
  Heather Turner, Jacob van Etten, David Firth, Ioannis Kosmidis. 2020.
  Modelling rankings in R: the PlackettLuce package. Comput Stat 35,
  1027--1057. \url{https://doi.org/10.1007/s00180-020-00959-3}
\end{itemize}

This article explains the Plackett-Luce model and its implementation in
R, as used by the ClimMob platform.

\begin{itemize}
\tightlist
\item
  Kauê de Sousa, Jacob van Etten, {[}\ldots{]} Matteo Dell'Acqua. 2021.
  Data-driven decentralized breeding increases prediction accuracy in a
  challenging crop production environment. Communications Biology 4,
  944. \url{https://doi.org/10.1038/s42003-021-02463-w}
\end{itemize}

This paper shows that tricot can be effectively combined with genomic
selection for highly accurate selection in challenging production
environments. Tested with durum wheat in Ethiopia, 3D-breeding doubled
prediction accuracy compared to conventional methods, identifying
genotypes with superior local adaptation across seasons to improve
breeding decisions.

\begin{itemize}
\tightlist
\item
  David Brown, Sytze de Bruin, Kauê de Sousa, {[}\ldots{]} Jacob van
  Etten. 2022. Rank-based data synthesis of common bean on-farm trials
  across four Central American countries. Crop Science.
  \url{https://doi.org/10.1002/csc2.20817}
\end{itemize}

This article provides an approach to combine data from different tricot
trials to obtain insights for regional analysis using on-farm data.

\begin{itemize}
\tightlist
\item
  Oladeji Emmanuel Alamu, Béla Teeken, et al.~2023. Stablishing the
  linkage between eba's instrumental and sensory descriptive profiles
  and their correlation with consumer preferences: implications for
  cassava breeding. Journal of the Science of Food and Agriculture.
  \url{https://doi.org/10.1002/jsfa.12518}
\end{itemize}

This article links tricot data to laboratory instrumental data to
understand consumers' preferences with implications for breeding
programs.

\begin{itemize}
\tightlist
\item
  Kauê de Sousa, David Brown, Jonathan Steinke, Jacob van Etten. 2023.
  gosset: An R package for analysis and synthesis of ranking data in
  agricultural experimentation. SoftwareX.
  \url{https://doi.org/10.1016/j.softx.2023.101402}
\end{itemize}

This paper introduces the gosset package used on ClimMob. It
demonstrates the package functionality using the case study of a
decentralized on-farm trial of common bean (Phaseolus vulgaris L.)
varieties in Nicaragua.

\begin{itemize}
\tightlist
\item
  Pieter Rutsaert, Jason Donovan, Harriet Mawia, Kauê de Sousa, Jacob
  van Etten. 2023. Future market segments for hybrid maize in East
  Africa. Market Intelligence Brief Series 2. Montpellier: CGIAR.
  \url{https://hdl.handle.net/10883/22467}
\end{itemize}

This paper introduces a novel approach to assess market demands in seed
systems using decentralized testing under the tricot approach.

\begin{itemize}
\tightlist
\item
  Carlos Quirós, Kauê de Sousa, {[}\ldots{]} Jacob van Etten. 2023.
  ClimMob: Software to Support Experimental Citizen Science in
  Agriculture. SSRN. \url{http://dx.doi.org/10.2139/ssrn.4463406}
\end{itemize}

This paper introduces the ClimMob software.

\begin{itemize}
\tightlist
\item
  Jacob van Etten, Kauê de Sousa, {[}\ldots{]} Hale Ann Tufan. 2023.
  Data-driven approaches can harness crop diversity to address
  heterogeneous needs for breeding products. PNAS 120 (14).
  \url{https://doi.org/10.1073/pnas.2205771120}
\end{itemize}

This paper brings a perspective on opportunities and challenges of
data-driven approaches for crop diversity management (genebanks and
breeding) in the context of agricultural research for sustainable
development in the Global South.

\begin{itemize}
\tightlist
\item
  Kauê de Sousa, Jacob van Etten, Rhys Manners, Erna Abidin,
  {[}\ldots{]} Mainassara Zaman-Allah. 2024. The tricot approach: an
  agile framework for decentralized on-farm testing supported by citizen
  science. A retrospective. Agronomy for Sustainable
  Development.https://doi.org/10.1007/s13593-023-00937-1
\end{itemize}

This paper reviews the development, validation, and large-scale
evolution of the decentralized, citizen-science-based tricot
method---highlighting its low cost, data validity and reliability at
scale, and its ability to capture socio-economic and climatic
heterogeneity to enhance crop variety decision-making in smallholder
agriculture.

\begin{itemize}
\tightlist
\item
  Ann Ritah Nanyonjo, Stephen Angudubo, Paula Iragaba,{[}\ldots{]}
  Robert Sezi Kawuki 2024. On-farm evaluation of cassava clones using
  the triadic comparison of technology options approach. Crop Science,
  64, 2679--2697. \url{https://doi.org/10.1002/csc2.21293}
\end{itemize}

This paper found that rainfall patterns shaped cassava clone performance
more than sociocultural factors, identifying several elite clones as
strong candidates for release.

\begin{itemize}
\tightlist
\item
  Martina Occelli, Jorge Sellare, Kauê de Sousa,{[}\ldots{]} Jacob van
  Etten. 2024. Group-based and citizen science on-farm variety selection
  approaches for bean growers in Central America. Agricultural
  Economics. \url{https://doi.org/10.1111/agec.12819}
\end{itemize}

This paper found that both tricot-PVS and traditional group-PVS
participation equally promoted on-farm varietal diversification among
Central American bean growers, but group-PVS more effectively enhanced
household food security---likely due to its positive impact on agronomic
management.

\begin{itemize}
\tightlist
\item
  Rachel Voss, Kauê de Sousa, Sognigbé N'Danikou,{[}\ldots{]} Maarten
  van Zonneveld. 2025. Citizen science informs demand-driven breeding of
  opportunity crops. Plants, People, Planet, 1--14.
  \url{https://doi.org/10.1002/ppp3.70035}
\end{itemize}

This paper used tricot trials with 2,063 farmers in West and East
Africa, showing that farmer preferences for leafy amaranth varied by
traits and demographics, guiding demand-driven breeding.

\begin{itemize}
\tightlist
\item
  K. Sharma, E. Atieno, J.Mugo, K. de Sousa, J. van Etten, S. Nyawade.
  2025. Understanding Farmer preferences through citizen Science:
  Insights on potato varieties in Nigeria. Journal of Agriculture and
  Food Research 23. \url{https://doi.org/10.1016/j.jafr.2025.102135}
\end{itemize}

This paper used tricot trials in Nigeria to reveal farmers preferred
three CIP potato genotypes for yield, resistance, and storability,
underscoring the value of participatory, data-driven breeding.

\begin{itemize}
\tightlist
\item
  Mabel Nabateregga, Hugo Dorado-Betancourt, Svein Solberg,{[}\ldots{]}
  Kauê de Sousa. 2025. Accuracy of farmer-generated yield estimations of
  common bean in decentralised on-farm trials in sub--Saharan Africa.
  Europen Journal of Agronomy.
  \url{https://doi.org/10.1016/j.eja.2025.127730}
\end{itemize}

This paper showed that farmer-generated yield estimates in Tanzanian
tricot bean trials closely matched technician and researcher
measurements, proving them accurate, scalable, and cost-effective for
breeding decisions.

\begin{itemize}
\tightlist
\item
  Geon Kang, Kauê de Sousa, Rhys Manners, Jacob van Etten,{[}\ldots{]}
  Stefanie Griebel.2025. Integrating environmental, socio-economic, and
  biological data in a farmer-led potato trial for enhanced varietal
  assessment in Rwanda. Experimental Agriculture. 61(18), pp 1-26.
  \url{https://doi:10.1017/S0014479725100100}
\end{itemize}

This paper found through tricot trials in Rwanda that farmers favored
older potato varieties, with yield and marketability---shaped by
temperature and income---driving adoption more than on-station
performance.

\begin{itemize}
\tightlist
\item
  Jason Donovan, Pieter Rutsaert, Harriet Mawia, Kauê de Sousa, Jacob
  van Etten. 2025. Farmers' preferences for the next generation of maize
  hybrids: application of product concept testing in Kenya and Uganda.
  \url{https://doi:10.1017/S001447972500002X}
\end{itemize}

This paper applied video-based product concept testing with 2,400
farmers in Kenya and Uganda, finding that ``Resilience,'' ``Drought
escape,'' and ``Intercropping'' seed concepts were most preferred,
highlighting opportunities for future maize breeding beyond
yield-focused traits.

\subsubsection{Training Videos}\label{videos}

All videos about ClimMob and tricot are on the ClimMob YouTube channel.
\url{https://www.youtube.com/channel/UCmqo4KCZwX8R-H4SNkXfuSA/playlists}
The videos are subdivided in three playing lists, as shown below.

\begin{enumerate}
\def\labelenumi{\arabic{enumi}.}
\item
  YouTube videos -- Introduction to tricot. These videos provide a
  general introduction to tricot as an on-farm testing approach. It
  explains the rationale and underlying concepts.
  \url{https://www.youtube.com/watch?v=uCZ9Hw5hubU&list=PLpT37wNlyZlRH2_K-sevTeLh2-bhYkY2h}
\item
  YouTube videos -- Introduction to ClimMob software. These videos
  provide a step-by-step explanation of how to set up an experiment on
  ClimMob.
  \url{https://www.youtube.com/watch?v=tkOwXG_Jyy4&list=PLpT37wNlyZlQNIrLdW7G91Xqaz_S3x_z0}
\item
  YouTube videos -- data analysis with R. The ClimMob platform provides
  trial-level analysis. For advanced analyses, R packages are available.
  These videos explain how to use them.
  \url{https://www.youtube.com/watch?v=pKYGjtwjagc&list=PLpT37wNlyZlS2QL67Qn-eLI8oETBr5sKm}
\end{enumerate}

\subsubsection{Costing Templates}\label{costing}

\subsubsection{Supporting Materials}\label{support_material}

\paragraph{Beans}\label{beans}

\href{/img/beans/Beans---Cooking-Time.jpg}{\pandocbounded{\includegraphics[keepaspectratio]{/img/beans/Beans---Cooking-Time.jpg}}
Click to open}

\paragraph{Cassava}\label{cassava}

\paragraph{Cocoa}\label{cocoa}

\paragraph{Cowpea}\label{cowpea}

\paragraph{Millet}\label{millet}

\paragraph{Potato}\label{potato}

\paragraph{Sorghum}\label{sorghum}

\paragraph{Soybean}\label{soybean}

\paragraph{Other tricot materials}\label{other-tricot-materials}

\end{document}
